\documentclass[11pt,a4paper]{report}

\usepackage[utf8]{inputenc}
\usepackage{titling}
\usepackage[german]{babel}
\usepackage[T1]{fontenc}
\usepackage{amsmath}
\usepackage{amsfonts}
\usepackage{amssymb}
\usepackage[left=3cm,right=2cm,top=2.5cm,bottom=2cm]{geometry}
\usepackage{graphicx}
\usepackage{fancyhdr}
\usepackage{color}
\usepackage[
colorlinks=true,
urlcolor=blue,
linkcolor=black
]{hyperref}
\pagestyle{fancy}

\lhead{Robin Seidel}
\chead{ak18b}
\rhead{13.12.2018}

\lfoot{}
\cfoot{\thepage}
\rfoot{}

\renewcommand{\headrulewidth}{0.4pt}
\renewcommand{\footrulewidth}{0.4pt}

\begin{document}
	\begin{titlepage}
		
		\pretitle{
			\vskip -3em
			\begin{figure}[h]
				\begin{center}
				
				\end{center}
			\end{figure}
			\begin{center}
				\vskip -2em
				\large{Wintersemester 2018/19\\Softwaretechnikpraktikum} \vskip 9em
				\rule{5in}{0.4pt}\par \vskip 0.5em
			}
			\posttitle{\par\rule{5in}{0.4pt} \vskip 4em
				\Large Gruppe: ak18b \vskip 1.5em
				\normalsize Betreuer: Benjamin Lucas Friedland, Michael Fritz\vskip 1em
				\normalsize Gruppenmitglieder: Alexander Zwisler, Leon Kamuf, Leon Rudolph, Maurice Eisenblätter, Maximilian Gläfcke, Robin Seidel, Sina Opitz, Steve Woywod
		\end{center}}
		
		\title{\textbf{\Huge App zur Inventarisierung von Unternehmenswerten}\vskip 0.5em \huge Releaseplan}
		\date{}
		\maketitle
	\end{titlepage}
	\setcounter{secnumdepth}{4}
	\setcounter{tocdepth}{4}
	\tableofcontents
	\thispagestyle{empty}
	\newpage
	\setcounter{page}{1}
	\renewcommand\thesection{\arabic{section}}

\section{Arbeitspaket 1 - Vorprojekt und Einarbeitung}
Release: 15.1.2019 \\
Das Vorprojekt bildet das wesentliche Grundgerüst der Anwendung. Es basiert auf dem Server- Client-Grundkonzept.\\
Es beinhaltet eine MariaDb Datenbank, eine REST-Schnittstelle sowie ein Frontend.\\
Weiterhin wird für das Frontend eine App-Shell entwickelt die das generelle Layout wiederspiegelt. Eine MariaDb mit REST Api wird auf dem Praktikumsserver eingerichtet.
Das Vorprojekt soll zur Einarbeitung in die verschiedenen Technologien dienen. 

\subsection{Umsetzung}
Folgende Muss-Ziele werden im Vorprojekt umgesetzt:\\
\\
- Login und Logout /LF0010/ und /LF0020/  \\
- Ändern Email Adresse /LF0110/ \\
- Änderung Passwort /LF0120/ \\
- Anzeigen aller erfassten Items /LF0210/ \\
- Sortieren von Items /LF0230/ \\
- Bearbeiten von Items /LF0330/ \\
- Entfernen von Items /LF0340/ \\
- Verlinkung von Items /LF0360/ \\
- Hinzufügen neuer Itemtypen /LF0410/ \\
- Bearbeitung von Itemtypen /LF0420/ \\
- Pflichtfelder von Itemtypen /LF0430/ \\
- Löschen von Itemtypen /LF0440/ \\
- Sicherheit der Anmeldedaten /LN0010/ \\
- Validieren von Benutzereingaben /LN0110/ \\
- Datensicherheit /LN0120/ \\
- Intuitive grafische Interaktion /LN0310/ \\
- Browsersupport /LN0410/ \\

\section{Arbeitspaket 2 - Funktionale Anforderungen }
Release: 29.1.2019 \\
\subsection{Umsetzung}
Im zweiten Arbeitspaket ist das Hauptthema die Administration von Unternehmen.\\
- Registrieren neuer Anwender /LF0030/ \\
- Wiederherstellung des Passworts /LF0040/ \\
- Suche /LF0220/ \\
- Eingabehilfe für Itemfelder /LF0350/ \\
- Löschen von Profilen /LF0520/ \\
- Hinzufügen von Unternehmen /LF0530/ \\
- Typenverwaltung /LF0540/ \\
- Globale Pflichtfelder /LF0550/ \\
\newpage

\section{Arbeitspaket 3 - Dokumentenverwaltung}
Release: 5.02.2019 \\
\subsection{Umsetzung}
Im dritten Arbeitspaket wird sich hauptsächlich mit der Dokumentenverwaltung beschäftigt. 
- Anhang von Dokumenten /LF0320/ \\
\section{Arbeitspaket 4 - Nicht Funktionale Anforderungen }
Release: 12.2.2019 \\
\subsection{Umsetzung}
- Im vorletzten Arbeitspaket werden eventuelle Restbestände der funktionalen und\\ nicht-funktionalen Anforderungen nachgebessert oder fertiggestellt. Ab dann werden Kann-Ziele umgesetzt.\\
- Mehrsprachigkeit /LF0610/ \\
- Itemtyp Vorlagen /LF0620/ \\
- Nutzerverwaltung /LF0660/ \\
\section{Arbeitspaket 5 - Optimierung und abschließende Tests}
Release 19.2.2019 \\
\subsection{Umsetzung}
Vor Abgabe des finalen Releases werden abschließende Tests durchgeführt. Optional können\\ Beta-User die Software testen um eventuelle Bugs und Ineffizienzen aufzuspüren. \\
- Statistiken für Itembestände /LF0630/ \\
- Benutzerhandbuch /LF0640/ \\
- Chat /LF0650/ \\



































\end{document}
