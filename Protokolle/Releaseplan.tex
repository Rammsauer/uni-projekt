\documentclass[11pt,a4paper]{report}

\usepackage[utf8]{inputenc}
\usepackage{titling}
\usepackage[german]{babel}
\usepackage[T1]{fontenc}
\usepackage{amsmath}
\usepackage{amsfonts}
\usepackage{amssymb}
\usepackage[left=3cm,right=2cm,top=2.5cm,bottom=2cm]{geometry}
\usepackage{graphicx}
\usepackage{fancyhdr}
\usepackage{color}
\usepackage[
colorlinks=true,
urlcolor=blue,
linkcolor=black
]{hyperref}
\pagestyle{fancy}

\lhead{Robin Seidel}
\chead{ak18b}
\rhead{13.12.2018}

\lfoot{}
\cfoot{\thepage}
\rfoot{}

\renewcommand{\headrulewidth}{0.4pt}
\renewcommand{\footrulewidth}{0.4pt}

\begin{document}
	\begin{titlepage}
		
		\pretitle{
			\vskip -3em
			\begin{figure}[h]
				\begin{center}
				\end{center}
			\end{figure}
			\begin{center}
				\vskip -2em
				\large{Wintersemester 2018/19\\Softwaretechnikpraktikum} \vskip 9em
				\rule{5in}{0.4pt}\par \vskip 0.5em
			}
			\posttitle{\par\rule{5in}{0.4pt} \vskip 4em
				\Large Gruppe: ak18b \vskip 1.5em
				\normalsize Betreuer: Benjamin Lucas Friedland, Michael Fritz\vskip 1em
				\normalsize Gruppenmitglieder: Alexander Zwisler, Leon Kamuf, Leon Rudolph, Maurice Eisenblätter, Maximilian Gläfcke, Robin Seidel, Sina Opitz, Steve Woywod
		\end{center}}
		
		\title{\textbf{\Huge App zur Inventarisierung von Unternehmenswerten}\vskip 0.5em \huge Releaseplan}
		\date{}
		\maketitle
	\end{titlepage}
	\setcounter{secnumdepth}{4}
	\setcounter{tocdepth}{4}
	\tableofcontents
	\thispagestyle{empty}
	\newpage
	\setcounter{page}{1}
	\renewcommand\thesection{\arabic{section}}

\section{Arbeitspaket 1 - Vorprojekt und Einarbeitung}

Das Vorprojekt bildet das wesentliche Grundgerüst der Anwendung. Es basiert auf dem Server- Client-Grundkonzept.\\
Es beinhaltet eine MariaDb Datenbank, eine REST-Schnittstelle sowie ein Fronend.\\
Weiterhin dient das Vorprojekt zur Einarbeitung in die verschiedenen Technologien. 
\subsection{Client}
- Der CLient ist eine Angular Applikation.\\
- Es sollen vollgende MUSS-Ziele eingebunden werden...
\subsection{Server}
- Der Server ist eine ExpressJS Application.\\
- Es sollen vollgende Dinge durch api bereits bereitgestellt werden...

\section{Arbeitpaket 2 - }
\subsection{placeholder}

\section{Arbeitspaket 3 - }
\subsection{placeholder}

\section{Arbeitspaket 3 -}
\subsection{placeholder}

\section{Arbeitspaket 4 - }
\subsection{placeholder}

\section{Arbeitspaket 5 - Optimierung und abschließende Tests}
\subsection{placeholder}






































\end{document}
